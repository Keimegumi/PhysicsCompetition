\documentclass[../main.tex]{subfiles}

\begin{document}

\chapter{向量和线性代数}
\begin{Obje}
    在这一章我们将把视角从简单的数投向一种更复杂的数学结构,完成以下目标:
    \begin{enumerate}
        \item 了解向量和其基本的计算
        \item 了解向量几何上的意义
        \item 学会简单的线性代数,并尝试应用
    \end{enumerate}
\end{Obje}
\section{二维向量}
\subsection{向量与向量运算}
\begin{definition}{二维向量}
    一个有序实数对称为一个\kwd{二维向量},我们用大写字母来表示向量
        \[
            \bm{U}=(u_1,u_2)
        \]
    数$u_1,u_2$称为向量的\kwd{分量},二维向量的全体称为\kwd{二维空间},记作$\mathbb{R}^2$

\end{definition}
\begin{definition}{向量的代数运算}
\begin{itemize}
    \item 用一个实数$c$\kwd{乘}向量$\bm{U}=(u_1,u_2)$得到的结果记为$c\bm{U}$,定义为用$c$乘上$\bm{U}$的各个分量
    \[
        c\bm{U}=(cu_1,cu_2)
    \]
    \item 向量$\bm{U}$和向量$\bm{V}$的\kwd{和}$\bm{U+V}$,定义为两个向量对应分量相加得到的向量:
    \[\bm{U+V}=(u_1+v_1,u_2+v_2)\]
\end{itemize}
\end{definition}
向量的数乘和相加满足通常的代数性质:
\begin{properties}
\begin{itemize}
    \item \kwd{交换律}:$\bm{U}\bm{V}=\bm{V}+\bm{U}$
    \item \kwd{结合律}: $(\bm{U}+\bm{V})+\bm{W}=\bm{U}+(\bm{V}+\bm{W})$
    \item \kwd{分配律}:$(a+b)\bm{U}=a\bm{U}+b\bm{V}$
    \item \kwd{分配律}:$c(\bm{U}+\bm{V})=c\bm{U}+c\bm{V}$
    \item \kwd{加法逆元}:$\bm{U}+(-\bm{U})=0$
\end{itemize}
\end{properties}
\notelabel 向量的另一种表示方法,是包裹在中括号之中,例如向量$\displaystyle (u_1,u_2)$可以被表表示为:
\[\begin{bmatrix} u_1\\u_2 \end{bmatrix}
\qquad \begin{bmatrix}
u_1&u_2
\end{bmatrix}^T
\]
下面我们开始讨论向量及向量运算在几何上的意义:
\subsection{向量的几何意义}
我们注意到任意一个向量$\bm{U}=\colv{x}{y}$都可以视为:
\[\bm{U}=(x,y)=x\colv{1}{0}+y\colv{0}{1}\]
我们可以将$\displaystyle \colvt{1}{0}$和$\displaystyle \colvt{0}{1}$分别看作是x轴和y轴的单位向量,那么可以将原本的向量 $\displaystyle \bm{U}$视为 $\displaystyle x-y$平面上的一个点。
\leftnote[-90pt]{
    在这里,我们可以将原本的 $\displaystyle \bm{U}$写作:
    \[
       x\colvt{1}{0}+y\colvt{0}{1} = \begin{bmatrix}
            1 &0\\
            0&1
        \end{bmatrix}
        \begin{bmatrix}
            x\\
            y    
        \end{bmatrix}
    \]  
     }  

\lefttikz[-20pt]{
    \begin{tikzpicture}[>=stealth, scale=1.2]
        \draw[->] (-0.5,0) -- (3,0) node[right] {\tiny $x$};
        \draw[->] (0,-0.5) -- (0,2.5) node[above] {\tiny $y$};
        % 向量U
        \draw[->, thick, zillTeal] (0,0) -- (2,0.5) node[right] {$U(3,2)$};
        % 向量V
        \draw[->, thick, zillTeal] (0,0) -- (0.8,1.8) node[above] {$V(1,5)$};
        % 辅助线
        \draw[dashed, gray] (2,0.5) -- (2.8,2.3);
        \draw[dashed, gray] (0.8,1.5) -- (2.8,2.3);
        % 合向量
        \draw[->, ultra thick, zillOrange] (0,0) -- (2.8,2.3) node[above right] {$U+V(4,7)$};
    \end{tikzpicture}
}{向量加法的平行四边形法则}

通过将向量视为平面中的点,运算具有下述几何解释:
\begin{enumerate}[label=(\alph*)]
    \item \kwd{数乘}:对于非零向量 $U$ 与实数 $c$,点 $cU$ 位于通过原点与 $U$ 的直线上。若 $c > 0$,方向相同;若 $c < 0$,方向相反。
    \item \kwd{加法}:若 $0, U, V$ 不在一条直线上,则 $0, U, U+V, V$ 构成一个平行四边形的四个顶点。
    \item \kwd{线段表达}:设 $0 \le c \le 1$,则点 $V + cU$ 位于从 $V$ 到 $V+U$ 的线段上。
\end{enumerate}

\subsection{线性组合与线性无关}

\begin{definition}{线性组合}
    向量 $U$ 与 $V$ 的一个 \kwd{线性组合} 是一个形如如下结构的向量:
    \[ aU + bV \]
    其中 $a$ 与 $b$ 是实数。
\end{definition}

\begin{example}[线性组合实例]
    证明 $U = (5, 3)$ 是 $(1, 1)$ 与 $(-1, 1)$ 的一个线性组合。
    \begin{solution}
        我们需要寻找 $a, b$ 使得 $a(1, 1) + b(-1, 1) = (5, 3)$。
        这产生方程组:
        \[ \begin{cases} a - b = 5 \\ a + b = 3 \end{cases} \]
        解得 $a=4, b=-1$。因此 $U = 4(1, 1) - (1, -1)$。
    \end{solution}
\end{example}

\begin{definition}{线性无关 (Linear Independence)}
    称向量 $U$ 与 $V$ \kwd{线性无关},如果它们的某个线性组合 $aU + bV = \mathbf{0}$ 仅在 $a=0, b=0$ 时成立。
    \par\medskip
    若存在不全为零的系数使得组合为零,则称它们为 \kwd{线性相关}。
\end{definition}

\leftnote[-70pt]{\meaninglabel\\ 如果两个向量线性相关,说明它们“共线”,即其中一个向量的信息可以由另一个完全替代。}

\begin{theorem}{线性无关的性质}
    给定位平面中两个线性无关的向量 $C$ 和 $D$,则 $\mathbb{R}^2$ 中的每一个向量 $U$ 都可以 \kwd{唯一} 地写成 $C$ 和 $D$ 的线性组合:
    \[ U = aC + bD \]
\end{theorem}

\subsection{线性函数}

\begin{definition}{线性函数}
    从 $\mathbb{R}^2$ 到实数集 $\mathbb{R}$ 的一个函数 $\ell: U \mapsto \ell(U)$ 称为线性的,如果:
    \begin{enumerate}
        \item $\ell(cU) = c\ell(U)$
        \item $\ell(U + V) = \ell(U) + \ell(V)$
    \end{enumerate}
\end{definition}

\begin{theorem}{线性函数的结构}
    从 $\mathbb{R}^2$ 到实数集 $\mathbb{R}$ 的每一个线性函数 $\ell$ 仅当它具有如下形式:
    \[ \ell(x, y) = px + qy \]
    其中 $p, q$ 是实数。
\end{theorem}
\subsection{平面的基向量和矩阵}
在前文我们提到,对于向量 $\displaystyle \bm{U}=(x,y)$,我们可以将其视为:
\[
\colv{x}{y}=x\colv{1}{0}+y\colv{0}{1}

\] 

\end{document}