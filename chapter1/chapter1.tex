\documentclass[../main.tex]{subfiles}

\begin{document}
\chapter{Preliminaries}
    \begin{Obje}
        在深入研究微积分学之前,我们需要总结一些基本的数学概念和符号,这一章的知识将会在后续章节中详细讨论
        \begin{enumerate}
            \item 了解基本的数学符号和数学概念
            \item 掌握函数和基本函数的性质
            \item 掌握解析几何的初等知识
            \item 了解向量的基本含义和运算法则
        \end{enumerate}
    \end{Obje}
\section{基本符号和概念}
    \begin{definition}{集合}
        集合是数学中最基本的概念,包含着一类具有相似性质的对象。集合中的对象成为元素,通常用小写字母表示,集合本身通常用大写元素字母表示。
    \end{definition}

    \begin{example}
        我们来考虑这样一个学校,其中老师和学生分别构成两个集合。设集合$T$表示所有的老师,集合$S$表示所有的学生:
        \[
            T=\{ \text{Chinese 1,Chinese 2,Math 1,Physics 1,English 1}\}
        \]
        \[
            S=\{\text{Tom,Charlie,Mike,Michael}\}
        \]
    \end{example}
    在学校中,我们可以看到老师和学生之间有一种特殊的关系,这种关系将两个集合联系起来。我们将这样的关系成为“二元关系”
    \begin{definition}{二元关系}
        设$A$和$B$是两个集合,$R$是$A$和$B$之间的一个二元关系,如果对于每一个$a\in A$,都存在一个$b\in B$,使得$(a,b)\in R$,则称$R$为从$A$到$B$的二元关系。
    \end{definition}
    \begin{example}
        在上面的学校例子中,我们可以定义一个二元关系$R$,表示老师与学生之间的指导关系。例如,Chinese 1老师指导Tom和Charlie,Math 1老师指导Mike和Michael。则有:
        \[
            R=\{(\text{Chinese 1}, \text{Tom}), (\text{Chinese 1}, \text{Charlie}), (\text{Math 1}, \text{Mike}), (\text{Math 1}, \text{Michael})\}
        \]
    \end{example}
    我们来考虑一种特殊的二元关系,对于每一个$a\in A$,都存在唯一的$b\in B$,使得$(a,b)\in R$,则称$R$为从$A$到$B$的映射,或者函数。
    \begin{definition}{函数}
        设$A$和$B$是两个集合,$f$是$A$到$B$的一个映射,如果对于每一个$a\in A$,都存在唯一的$b\in B$,使得$(a,b)\in f$,则称$f$为从$A$到$B$的函数,记作$f:A\to B$,并称$b=f(a)$为$a$在$f$下的像。
    \end{definition}
    \begin{example}
        在上面的学校例子中,我们可以定义一个函数$f$,表示每个学生对应的指导老师。例如,Tom和Charlie的指导老师是Chinese 1,Mike和Michael的指导老师是Math 1。则有:
        \[
            f(\text{Tom})=\text{Chinese 1}, \quad f(\text{Charlie})=\text{Chinese 1}, \quad f(\text{Mike})=\text{Math 1}, \quad f(\text{Michael})=\text{Math 1}
        \]
    \end{example}
    
    在这里,对于$A$和$B$之间的函数$f:A\to B$,我们称$A$为定义域,$B$为陪域,$f(A)$为$f$的值域。
    
    在实际生活中,我们常常会发现一个变量的值依赖于另一个变量的值:
        \begin{itemize}
            \item 水的沸点依赖于海拔的高度
            \item 人口的增长依赖于出生率
        \end{itemize}
    
    在上述的情形中,一个变量的值常常取决于另外一个变量的值,水的沸点$b$依赖于海拔高度$h$。我们可以利用函数来表达他们,将$b$称之为\textbf{因变量},$h$称之为\textbf{自变量}
    \begin{example}\textbf
        已知圆的面积$A$依赖于圆的半径$r$,则可以用函数表示为$A=f(r)$
        \[
            A=\pi r^2
        \]
    这个过程中我们可以知道:
        \begin{itemize}
            \item 函数的值域为所有非负实数。定义域是所有可能的半径的集合
            \item 因变量是圆的面积$A$,自变量是圆的半径$r$
        \end{itemize}
    \end{example}

    \leftnote[-90pt]{将函数想象为一个机器也有助于理解这一概:将自变量input到机器中,经过函数(机器)的处理,会output出唯一一个值——因变量。}
    

    最后,我们需要了解一些详细的数学符号:
    \begin{itemize}
        \item 实数集$\mathbb{R}$:表示所有实数的集合
        \item 自然数集$\mathbb{N}$:表示所有自然数的集合
        \item 整数集$\mathbb{Z}$:表示所有整数的集合
        \item 有理数集$\mathbb{Q}$:表示所有有理数的集合
    \end{itemize}
\begin{exercises}{1.1}    
    % 这里写题目的要求(通常是斜体)
    \noindent\textit{在问题1-3中,请根据题目描述设定集合,并显式地写出二元关系 $R$ 中的部分元素。}
    \begin{exerciseList}
        \item 将一些水果设定为集合 $F = \{\text{苹果, 香蕉, 葡萄}\}$,将一些颜色设定为集合 $C = \{\text{红色, 黄色, 紫色, 绿色}\}$。定义一个“具有该颜色”的二元关系 $R$。
        \item 将国家设定为集合 $N = \{\text{中国, 日本, 英国, 法国}\}$,将洲设定为集合 $S = \{\text{亚洲, 欧洲, 非洲}\}$。定义一个“属于该洲”的二元关系 $R$。
        \item 将一些实数设定为集合 $A = \{-2, -1, 0, 1, 2\}$,集合 $B = \{0, 1, 4\}$。定义一个“平方关系” $R$,即 $(a, b) \in R$ 当且仅当 $a^2 = b$。
    \end{exerciseList}

    \noindent\textit{在问题4-6中,判断所给出的二元关系是否构成“从 $A$ 到 $B$ 的函数”,并说明理由。}
    \begin{exerciseList}
        \item 集合 $A$ 为平面上所有的三角形,集合 $B$ 为所有的正实数。二元关系 $f$ 定义为:每个三角形对应它的面积。
        \item 集合 $A$ 为所有的实数 $\mathbb{R}$,集合 $B$ 也是 $\mathbb{R}$。二元关系 $f$ 定义为:$f = \{(x, y) \mid y^2 = x\}$。
        \item 集合 $A$ 为某物理实验室中的所有温度计,集合 $B$ 为当前室内的温度值。二元关系 $f$ 为:每支温度计显示的读数。
    \end{exerciseList}

    \noindent\textit{在问题7-9中,识别下列物理公式中的自变量与因变量。}
    \begin{exerciseList}
        \item 欧姆定律 $I = \frac{U}{R}$。当电阻 $R$ 固定时,电流 $I$ 随电压 $U$ 变化。
        \item 匀加速运动的速度公式 $v = v_0 + at$。其中 $v_0$ 和 $a$ 是常数。
        \item 理想气体的压强公式 $P = \frac{nRT}{V}$。假设温度 $T$ 和物质的量 $n$ 固定。
    \end{exerciseList}
\end{exercises}

\section{函数的性质和表达}
\subsection{函数的表示法}
函数有以下四种表示方法:
 \begin{itemize}
    \item 描述法  (用语言来描述)
    \item 数值法  (用表格列出函数值)
    \item 图像法  (用函数图像)
    \item 代数法  (用显示方程)
 \end{itemize}
\subsection{函数的奇偶性}
\begin{definition}{函数的奇偶性}
    如果$y=f(x)$是:
    \[
    \begin{aligned}
          &x\text{的奇函数,如果}f(x)+f(-x)=0\\
          &x\text{的偶函数,如果}f(x)=f(-x)
    \end{aligned}
    \]
    对于函数的整个区间成立
\end{definition}
\begin{example}
    判断下列函数的奇偶性:
    \begin{itemize}
        \item $f(x)=x^3+2x$,因为$f(-x)=-x^3-2x=-f(x)$,所以$f(x)$是奇函数
        \item $f(x)=x^2+3$,因为$f(-x)=(-x)^2+3=x^2+3=f(x)$,所以$f(x)$是偶函数
        \item $f(x)=x^2+x$,因为$f(-x)=(-x)^2+(-x)=x^2-x\neq f(x)$且$f(-x)\neq -f(x)$,所以$f(x)$既不是奇函数也不是偶函数
    \end{itemize}
\end{example}
\subsection{函数的单调性}
\begin{definition}{单调性和单调区间}
    设函数$y=f(x)$在区间$I$上有定义:
    \begin{itemize}
         \item 如果对于任意$x_1,x_2\in I$,当$x_1<x_2$时,均有$f(x_1)<f(x_2)$,则称函数$f(x)$在区间上是增函数。
         \item 如果对于任意$x_1,x_2\in I$,当$x_1<x_2$时,均有$f(x_1)>f(x_2)$,则称函数$f(x)$在区间上是减函数。
    \end{itemize}
将区间$I$成为函数$f(x)$的单调区间,函数$f(x)$在区间$I$上具有单调性。
\end{definition}
\begin{example}
    判断下列函数的单调性
    
    \begin{itemize}
        \item $f(x)=3x+2$,该函数在$\mathbb{R}$上单调递增
        \item $f(x)=2^x+3$,该函数在$\mathbb{R}$上单调递增
        \item $f(x)=x^2-3x+2$,该函数在$(-\infty, \frac{3}{2})$上单调递减,在$(\frac{3}{2}, +\infty)$上单调递增
    \end{itemize}

\end{example}

\subsection{函数的变换}
\subsubsection{平移变换}
\textbf{垂直平移图形}:
    \begin{itemize}
        \item 往上平移函数$y=f(x)$的图形,加一正常数到公式的右边。
        \item 往下平移函数$y=f(x)$的图形,加一负常数到公示的右边。
    \end{itemize}
\begin{example}
在公式$y=x^2$的基础上右端加一常数$1$就得到$y=x^2+1$,把图形往上移位了一个单位。

在公式$y=x^2$的基础上右端加$-2$就得到$y=x^2-2$,把图形下移了两个单位。
\end{example}

\textbf{水平平移图形}
    \begin{itemize}
        \item 往右平移函数$y=f(x)$的图形,在$x$上加一个负常数.
        \item 往左平移函数$y=f(x)$的图形,在$x$上加一个正常数.
    \end{itemize}

\begin{example}
在公式$y=x^2$的基础上把$x$替换成$x-3$就得到$y=(x-3)^2$,把图形往右移了三个单位。

在公式$y=x^2$的基础上把$x$替换成$x+2$就得到$y=(x+2)^2$,把图形往左移了两个单位。
\end{example}

\subsubsection{对称变换}
\begin{itemize}
    \item\textbf{关于$x$轴对称}:把函数$y=f(x)$的图形关于$x$轴对称,只需把公式右端的$y$替换成$-y$,即得到$y=-f(x)$。
    \item\textbf{关于$y$轴对称}:把函数$y=f(x)$的图形关于$y$轴对称,只需把公式中的$x$替换成$-x$,即得到$y=f(-x)$。
\end{itemize}
\begin{example}
    把函数$y=x^3$的图形关于$x$轴对称,得到$y=-x^3$。

    把函数$y=x^3$的图形关于$y$轴对称,得到$y=(-x)^3=-x^3$。
\end{example}

\subsection{复合函数和反函数}
\begin{definition}{复合函数}
设存在两个函数分别为:$f:A\to B$和$g:B\to C$,则可以定义复合函数如下:
\[
    (g\circ f)(x)=g(f(x))
\]
其中,$g\circ f:A \to C$成为$f$和$g$的复合函数
\end{definition}
\leftnote[-70pt]{
    可以将$f$和$g$分别看作是一个机器,那么复合函数$g\circ f$就是一个流水线,原材料先输入到$f$机器中然后output再作为$g$机器的input。
}
\begin{example}
在恒温条件下,理想气体的压强$P$和体积$V$之间的关系为$PV=C$,其中$C$为常数。如图所示,气缸的活塞可以上下移动以改变气体的体积$V$。试求出压强$P$与活塞高度$h$之间的函数关系。

\begin{solution}
设气缸的横截面积为$S$,则体积$V$与活塞高度$h$之间的关系为:
\[
     V=f(h)=Sh
\]
则复合函数$P(h)=P(V(h))=g\circ f$为:
\[P(h)=\frac{C}{f(h)}=\frac{C}{Sh}\]
\end{solution}
\end{example}

\begin{definition}{反函数}
    设存在两个函数$f:A\to B$和$g:B \to A$,如果对于任意的$a \in A$和$b \in B$,满足:
     \[
     (g\circ f)a=a,\quad(f\circ g)b=b
     \]
     我们称$f$和$g$互为反函数。
\end{definition}

\end{document}