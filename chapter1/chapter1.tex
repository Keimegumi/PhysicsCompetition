\documentclass[../main.tex]{subfiles}

\begin{document}
\chapter{Introduction to ZillStyleBook}
\begin{objectives}
    \begin{enumerate}
        \item 了解Zillstyle的使用方法
        \item 了解环境的大概配置
    \end{enumerate}
\end{objectives}

\section{章节标题}
    \begin{definition}{定理名称}
    Zillstyle是参考David.D.Zill的微分方程一书而编写的latex文档模版,来让受众可以更轻松的进行文档排版
    \end{definition}

    \begin{properties}
        目前文档配置的环境有:
        \begin{enumerate}
            \item 目录前言(objectives):让用户可以设置每一章的使用目标
            \item 定义(definition):让用户可以完善环境
            \item 性质(properties):提供性质环境书写性质
            \item 定理(theorem):帮助用户配置定理环境
        \end{enumerate}
        
    \end{properties}

    \begin{theorem}{定理名称}
        定理环境非常的清晰,十分好用个人感觉
    \end{theorem}
    
    \begin{example}
        这是一个例子
    \end{example}

    \begin{exercises}{1}
    
    % 这里写题目的要求(通常是斜体)
    \noindent\textit{In Problems 1--4, determine the order of the given differential equation.}

    % 开启题目列表
    \begin{exerciseList}
        \item $y'' + 5y' + 6y = 0$
        \item $\displaystyle \frac{d^4y}{dx^4} + \frac{d^2y}{dx^2} = \cos x$
        \item $(1-x)y'' - 4xy' + 5y = \cos x$
        \item $x^3 y''' - x^2 y'' + 4xy' - 3y = 0$
    \end{exerciseList}
    \end{exercises}
\end{document}