\documentclass[../main.tex]{subfiles}

\begin{document}
\chapter{Preliminaries}
    \begin{Obje}
        在深入研究微积分学之前,我们需要总结一些基本的数学概念和符号,这一章的知识将会在后续章节中详细讨论
        \begin{enumerate}
            \item 了解基本的数学符号和数学概念
            \item 掌握函数和基本函数的性质
            \item 掌握解析几何的初等知识
            \item 了解向量的基本含义和运算法则
        \end{enumerate}
    \end{Obje}
\section{基本符号和概念}
    \begin{definition}{集合}
        集合是数学中最基本的概念,包含着一类具有相似性质的对象。集合中的对象成为元素,通常用小写字母表示,集合本身通常用大写元素字母表示。
    \end{definition}

    \begin{example}
        我们来考虑这样一个学校,其中老师和学生分别构成两个集合。设集合$T$表示所有的老师,集合$S$表示所有的学生:
        \[
            T=\{ \text{Chinese 1,Chinese 2,Math 1,Physics 1,English 1}\}
        \]
        \[
            S=\{\text{Tom,Charlie,Mike,Michael}\}
        \]
    \end{example}
    在学校中,我们可以看到老师和学生之间有一种特殊的关系,这种关系将两个集合联系起来。我们将这样的关系成为“二元关系”
    \begin{definition}{二元关系}
        设$A$和$B$是两个集合,$R$是$A$和$B$之间的一个二元关系,如果对于每一个$a\in A$,都存在一个$b\in B$,使得$(a,b)\in R$,则称$R$为从$A$到$B$的二元关系。
    \end{definition}
    \begin{example}
        在上面的学校例子中,我们可以定义一个二元关系$R$,表示老师与学生之间的指导关系。例如,Chinese 1老师指导Tom和Charlie,Math 1老师指导Mike和Michael。则有:
        \[
            R=\{(\text{Chinese 1}, \text{Tom}), (\text{Chinese 1}, \text{Charlie}), (\text{Math 1}, \text{Mike}), (\text{Math 1}, \text{Michael})\}
        \]
    \end{example}
    我们来考虑一种特殊的二元关系,对于每一个$a\in A$,都存在唯一的$b\in B$,使得$(a,b)\in R$,则称$R$为从$A$到$B$的映射,或者函数。
    \begin{definition}{函数}
        设$A$和$B$是两个集合,$f$是$A$到$B$的一个映射,如果对于每一个$a\in A$,都存在唯一的$b\in B$,使得$(a,b)\in f$,则称$f$为从$A$到$B$的函数,记作$f:A\to B$,并称$b=f(a)$为$a$在$f$下的像。
    \end{definition}
    \begin{example}
        在上面的学校例子中,我们可以定义一个函数$f$,表示每个学生对应的指导老师。例如,Tom和Charlie的指导老师是Chinese 1,Mike和Michael的指导老师是Math 1。则有:
        \[
            f(\text{Tom})=\text{Chinese 1}, \quad f(\text{Charlie})=\text{Chinese 1}, \quad f(\text{Mike})=\text{Math 1}, \quad f(\text{Michael})=\text{Math 1}
        \]
    \end{example}
    
    在这里,对于$A$和$B$之间的函数$f:A\to B$,我们称$A$为定义域,$B$为陪域,$f(A)$为$f$的值域。
    
    在实际生活中,我们常常会发现一个变量的值依赖于另一个变量的值:
        \begin{itemize}
            \item 水的沸点依赖于海拔的高度
            \item 人口的增长依赖于出生率
        \end{itemize}
    
    在上述的情形中,一个变量的值常常取决于另外一个变量的值,水的沸点$b$依赖于海拔高度$h$。我们可以利用函数来表达他们,将$b$称之为\textbf{因变量},$h$称之为\textbf{自变量}
    \begin{example}\textbf
        已知圆的面积$A$依赖于圆的半径$r$,则可以用函数表示为$A=f(r)$
        \[
            A=\pi r^2
        \]
    这个过程中我们可以知道:
        \begin{itemize}
            \item 函数的值域为所有非负实数。定义域是所有可能的半径的集合
            \item 因变量是圆的面积$A$,自变量是圆的半径$r$
        \end{itemize}
    \end{example}

    \leftnote{将函数想象为一个机器也有助于理解这一概:将自变量input到机器中,经过函数(机器)的处理,会output出唯一一个值——因变量。}
    

    最后,我们需要了解一些详细的数学符号:
    \begin{itemize}
        \item 实数集$\mathbb{R}$:表示所有实数的集合
        \item 自然数集$\mathbb{N}$:表示所有自然数的集合
        \item 整数集$\mathbb{Z}$:表示所有整数的集合
        \item 有理数集$\mathbb{Q}$:表示所有有理数的集合
    \end{itemize}
\begin{exercises}{1.1}    
    % 这里写题目的要求(通常是斜体)
    \noindent\textit{在问题1-3中,请根据题目描述设定集合,并显式地写出二元关系 $R$ 中的部分元素。}
    \begin{exerciseList}
        \item 将一些水果设定为集合 $F = \{\text{苹果, 香蕉, 葡萄}\}$,将一些颜色设定为集合 $C = \{\text{红色, 黄色, 紫色, 绿色}\}$。定义一个“具有该颜色”的二元关系 $R$。
        \item 将国家设定为集合 $N = \{\text{中国, 日本, 英国, 法国}\}$,将洲设定为集合 $S = \{\text{亚洲, 欧洲, 非洲}\}$。定义一个“属于该洲”的二元关系 $R$。
        \item 将一些实数设定为集合 $A = \{-2, -1, 0, 1, 2\}$,集合 $B = \{0, 1, 4\}$。定义一个“平方关系” $R$,即 $(a, b) \in R$ 当且仅当 $a^2 = b$。
    \end{exerciseList}

    \noindent\textit{在问题4-6中,判断所给出的二元关系是否构成“从 $A$ 到 $B$ 的函数”,并说明理由。}
    \begin{exerciseList}
        \item 集合 $A$ 为平面上所有的三角形,集合 $B$ 为所有的正实数。二元关系 $f$ 定义为:每个三角形对应它的面积。
        \item 集合 $A$ 为所有的实数 $\mathbb{R}$,集合 $B$ 也是 $\mathbb{R}$。二元关系 $f$ 定义为:$f = \{(x, y) \mid y^2 = x\}$。
        \item 集合 $A$ 为某物理实验室中的所有温度计,集合 $B$ 为当前室内的温度值。二元关系 $f$ 为:每支温度计显示的读数。
    \end{exerciseList}

    \noindent\textit{在问题7-9中,识别下列物理公式中的自变量与因变量。}
    \begin{exerciseList}
        \item 欧姆定律 $I = \frac{U}{R}$。当电阻 $R$ 固定时,电流 $I$ 随电压 $U$ 变化。
        \item 匀加速运动的速度公式 $v = v_0 + at$。其中 $v_0$ 和 $a$ 是常数。
        \item 理想气体的压强公式 $P = \frac{nRT}{V}$。假设温度 $T$ 和物质的量 $n$ 固定。
    \end{exerciseList}
\end{exercises}

\section{函数的性质和表达}
\subsection{函数的表示法}
函数有以下四种表示方法:
 \begin{itemize}
    \item 描述法  (用语言来描述)
    \item 数值法  (用表格列出函数值)
    \item 图像法  (用函数图像)
    \item 代数法  (用显示方程)
 \end{itemize}
\subsection{函数的奇偶性}
\begin{definition}{函数的奇偶性}
    如果$y=f(x)$是:
    \[
    \begin{aligned}
          &x\text{的奇函数,如果}f(x)+f(-x)=0\\
          &x\text{的偶函数,如果}f(x)=f(-x)
    \end{aligned}
    \]
    对于函数的整个区间成立
\end{definition}
\begin{example}
    判断下列函数的奇偶性:
    \begin{itemize}
        \item $f(x)=x^3+2x$,因为$f(-x)=-x^3-2x=-f(x)$,所以$f(x)$是奇函数
        \item $f(x)=x^2+3$,因为$f(-x)=(-x)^2+3=x^2+3=f(x)$,所以$f(x)$是偶函数
        \item $f(x)=x^2+x$,因为$f(-x)=(-x)^2+(-x)=x^2-x\neq f(x)$且$f(-x)\neq -f(x)$,所以$f(x)$既不是奇函数也不是偶函数
    \end{itemize}
\end{example}
\subsection{函数的单调性}
\begin{definition}{单调性和单调区间}
    设函数$y=f(x)$在区间$I$上有定义:
    \begin{itemize}
         \item 如果对于任意$x_1,x_2\in I$,当$x_1<x_2$时,均有$f(x_1)<f(x_2)$,则称函数$f(x)$在区间上是增函数。
         \item 如果对于任意$x_1,x_2\in I$,当$x_1<x_2$时,均有$f(x_1)>f(x_2)$,则称函数$f(x)$在区间上是减函数。
    \end{itemize}
将区间$I$成为函数$f(x)$的单调区间,函数$f(x)$在区间$I$上具有单调性。
\end{definition}
\begin{example}
    判断下列函数的单调性
    
    \begin{itemize}
        \item $f(x)=3x+2$,该函数在$\mathbb{R}$上单调递增
        \item $f(x)=2^x+3$,该函数在$\mathbb{R}$上单调递增
        \item $f(x)=x^2-3x+2$,该函数在$(-\infty, \frac{3}{2})$上单调递减,在$(\frac{3}{2}, +\infty)$上单调递增
    \end{itemize}

\end{example}

\subsection{函数的变换}
\subsubsection{平移变换}
\textbf{垂直平移图形}:
    \begin{itemize}
        \item 往上平移函数$y=f(x)$的图形,加一正常数到公式的右边。
        \item 往下平移函数$y=f(x)$的图形,加一负常数到公示的右边。
    \end{itemize}
\begin{example}
在公式$y=x^2$的基础上右端加一常数$1$就得到$y=x^2+1$,把图形往上移位了一个单位。

在公式$y=x^2$的基础上右端加$-2$就得到$y=x^2-2$,把图形下移了两个单位。
\end{example}

\textbf{水平平移图形}
    \begin{itemize}
        \item 往右平移函数$y=f(x)$的图形,在$x$上加一个负常数.
        \item 往左平移函数$y=f(x)$的图形,在$x$上加一个正常数.
    \end{itemize}

\begin{example}
在公式$y=x^2$的基础上把$x$替换成$x-3$就得到$y=(x-3)^2$,把图形往右移了三个单位。

在公式$y=x^2$的基础上把$x$替换成$x+2$就得到$y=(x+2)^2$,把图形往左移了两个单位。
\end{example}

\subsubsection{对称变换}
\begin{itemize}
    \item\textbf{关于$x$轴对称}:把函数$y=f(x)$的图形关于$x$轴对称,只需把公式右端的$y$替换成$-y$,即得到$y=-f(x)$。
    \item\textbf{关于$y$轴对称}:把函数$y=f(x)$的图形关于$y$轴对称,只需把公式中的$x$替换成$-x$,即得到$y=f(-x)$。
\end{itemize}
\begin{example}
    把函数$y=x^3$的图形关于$x$轴对称,得到$y=-x^3$。

    把函数$y=x^3$的图形关于$y$轴对称,得到$y=(-x)^3=-x^3$。
\end{example}

\subsection{复合函数和反函数}
\begin{definition}{复合函数}
设存在两个函数分别为:$f:A\to B$和$g:B\to C$,则可以定义复合函数如下:
\[
    (g\circ f)(x)=g(f(x))
\]
其中,$g\circ f:A \to C$成为$f$和$g$的复合函数
\end{definition}
\leftnote[-70pt]{
    可以将$f$和$g$分别看作是一个机器,那么复合函数$g\circ f$就是一个流水线,原材料先输入到$f$机器中然后output再作为$g$机器的input。
}
\begin{example}
在恒温条件下,理想气体的压强$P$和体积$V$之间的关系为$PV=C$,其中$C$为常数。如图所示,气缸的活塞可以上下移动以改变气体的体积$V$。试求出压强$P$与活塞高度$h$之间的函数关系。

\begin{solution}
设气缸的横截面积为$S$,则体积$V$与活塞高度$h$之间的关系为:
\[
     V=f(h)=Sh
\]
则复合函数$P(h)=P(V(h))=g\circ f$为:
\[P(h)=\frac{C}{f(h)}=\frac{C}{Sh}\]
\end{solution}
\end{example}

\begin{definition}{反函数}
    设存在两个函数$f:A\to B$和$g:B \to A$,如果对于任意的$a \in A$和$b \in B$,满足:
     \[
     (g\circ f)a=a,\quad(f\circ g)b=b
     \]
     我们称$f$和$g$互为反函数,符号记作$g(x)=f^{-1}(x)$
\end{definition}
\leftnote[-70pt]{
    这里我们需要格外注意反函数的定义域和值域问题,事实上一个具有反函数的函数一定是单射的
}
\begin{example}[自由落体的“逆向思考”]
已知自由落体下落的高度$h(t)$和时间$t$的函数关系为:
\[
h(t)=\frac{1}{2}gt^2\]
在这个过程中,时间$t$和下落高度的函数关系为:
\[
t=f(h)=\sqrt{\frac{2h}{g}}\]
则函数$h(t)$和函数$f(h)$是一对反函数。其中:
\begin{itemize}
    \item $h(t)$的定义域是下落的时间(非负实数),值域是下落的高度(非负实数)
    \item $f(h)$的定义域是下落的高度(非负实数),值域是下落的时间(非负实数)
\end{itemize}
\end{example}
\begin{properties}
    \begin{enumerate}
        \item \textbf{图像对称}:图像$y=f(x)$与其反函数 $y=f^{-1}(x)$ 的图像,关于直线 $y=x$ 对称
        \item \textbf{奇偶性关联}:
            \begin{itemize}[topsep=-\baselineskip, nosep]
                                \item 若 $f(x)$ 是奇函数且存在反函数,则 $f^{-1}(x)$ 同样是一个奇函数
                                \item 若 $f(x)$ 是一个偶函数,则在其对称区间上不存在反函数
            \end{itemize}
        \item \textbf{复合函数性质}:对于互为反函数的 $f$ 和 $f^{-1}$ ,有 $f\circ f^{-1}=f^{-1}\circ f=\text{id}$,id表示恒等函数
    \end{enumerate}
\end{properties}
\begin{exercises}{1.2}

    \noindent\textit{在问题 1-5 中,探讨函数的奇偶性与单调性及其物理含义。}
    \begin{exerciseList}
        \item 判断函数 $f(x) = \frac{1}{x} + x^3$ 的奇偶性,并说明其在 $(0, +\infty)$ 上的单调性。
        \item \textbf{弹簧弹性势能}:公式为 $E_p(x) = \frac{1}{2}kx^2$。若考虑位移 $x$ 可正可负(代表压缩或拉伸):
            \begin{enumerate}
                \item 证明 $E_p(x)$ 是偶函数。
                \item 解释这个结论的物理意义(提示:关于平衡位置对称)。
            \end{enumerate}
        \item 已知函数 $f(x)$ 是定义在 $\mathbb{R}$ 上的奇函数。若当 $x \in (0, 1)$ 时 $f(x) = 2x + 1$,求 $f(x)$ 在 $(-1, 0)$ 上的解析式。
        \item 观察函数 $f(x) = \frac{1}{x^2+1}$。通过分析 $x^2+1$ 的性质,判断 $f(x)$ 的最大值及对应的 $x$ 值。
        \item 物理中的**静电力**公式为 $F(r) = k\frac{q_1q_2}{r^2}$。若电荷量固定,分析该函数在 $r \in (0, +\infty)$ 上的单调性。
    \end{exerciseList}

    \noindent\textit{在问题 6-11 中,通过平移与对称变换处理物理图像。}
    \begin{exerciseList}
        \item 将抛物线 $y = x^2$ 整体向左平移 2 个单位,再向上平移 3 个单位,写出新的函数解析式。
        \item \textbf{高度偏移}:一架无人机从海拔 $100\text{m}$ 的平台上起飞,其相对于平台的高度为 $h(t) = 5t^2$。请写出无人机相对于海平面高度 $H(t)$ 的函数。这属于哪种平移?
        \item 考虑正弦波形 $y = \sin x$。若将其向右平移 $\pi/2$ 个单位,得到的解析式是什么?它与 $\cos x$ 的图像有什么关系?
        \item \textbf{波动预演}:若波形 $f(x) = x^2$ 以 $v = \qty{2}{m/s}$ 的速度向右传播,写出 $t = \qty{5}{s}$ 时的波形解析式(提示:$x \to x-vt$)。
        \item 证明:任何函数 $y = f(x)$ 关于 $x$ 轴对称后的图像对应的函数一定是 $-f(x)$。
        \item 将函数 $y = \sqrt{x}$ 关于 $y$ 轴对称,并说明所得新函数的定义域。
    \end{exerciseList}

    \noindent\textit{在问题 12-15 中,练习复合函数的拆解与代入。}
    \begin{exerciseList}
        \item 已知 $f(x) = 2x - 1$,$g(x) = x^2$。分别计算复合函数 $f(g(x))$ 和 $g(f(x))$。它们相等吗?
        \item \textbf{功率链}:一台电动机的输出功率 $P = I^2 R$。若电流随时间变化为 $I(t) = I_0 \sin(\omega t)$,请写出功率随时间变化的复合函数 $P(t)$。
        \item 设 $u(x) = 1 - x$,$y(u) = \frac{1}{1-u}$。求 $y$ 关于 $x$ 的复合函数,并简化结果。
        \item 物理中,圆周运动的向心加速度 $a = \frac{v^2}{r}$。若速度随时间线性增加 $v = v_0 + kt$,写出 $a$ 随时间 $t$ 变化的函数。
    \end{exerciseList}

    \noindent\textit{在问题 16-20 中,求解反函数并探讨其物理应用。}
    \begin{exerciseList}
        \item 求函数 $f(x) = \frac{2x+1}{x-1}$ 的反函数。
        \item \textbf{动量与速度}:动量公式为 $p = mv$。若已知动量 $p$,求速度 $v$ 关于 $p$ 的反函数。
        \item \textbf{折射定律}:已知 $y = \sin \theta$。在物理实验中,我们测得了 $y$ 的值。请问如何表示 $\theta$?(注:只需说明这是反函数关系)。
        \item 考虑函数 $f(x) = x^2$。为什么在整个实数集 $\mathbb{R}$ 上它没有反函数?要使其拥有反函数,我们需要对 $x$ 做什么样的限制?
        \item \textbf{综合思考题}:证明如果一个函数 $f(x)$ 既是奇函数又存在反函数 $f^{-1}(x)$,则其反函数也一定是奇函数。(提示:从 $f(f^{-1}(-x))$ 出发)。
    \end{exerciseList}

\end{exercises}
\section{基本初等函数}
\begin{obje}
    我们要花一些时间来研究几个基本函数的性质,这些函数将在实际问题中反复用到:
    \begin{enumerate}
        \item 了解三角函数和反三角函数的定义和变换
        \item 了解指数函数和对数函数的定义和变换
        \item 了解幂函数的基本运算性质
    \end{enumerate}
\end{obje}
\subsection{三角函数和反三角函数}
我们先回忆中学视角下的三角函数。对于一个直角三角形,设其中一个锐角为$\theta$ ,其对边(opposite)、邻边(adjacent)和斜边(hypotenuse)分别为$a,b,c$
那么有:
\[
\sin\theta=\frac{a}{c}\quad \cos\theta=\frac{b}{c}\quad \tan\theta =\frac{\sin \theta}{\cos\theta}=\frac{a}{b} \qquad\theta \in (0^\circ,90^\circ)
\]
现在,我们需要将这一个函数的定义域扩展到所有角度,单位元的三角函数定义如下:
\begin{definition}{单位圆下的三角函数}
    单位圆是以原点为中心,半径为1的圆。任意角度$\theta$对应于单位圆上的一点,三角函数的定义如下
    \begin{itemize}
        \item 正弦($\sin$):一个角$\theta$对应的点的$y$坐标
        \item 余弦($\cos$):一个角$\theta$对应的点的$x$坐标
        \item 正切($\tan$):正弦和余弦的比值$\displaystyle \tan{\theta}=\frac{\sin\theta}{\cos\theta}$
    \end{itemize}
\end{definition}

\lefttikz[-110pt]{
    \begin{tikzpicture}[>=stealth, scale=1.7]
        % 坐标轴:标签全部加了 $ $ 保护
        \draw[->] (-1.2,0) -- (1.2,0) node[right] {$x$};
        \draw[->] (0,-1.2) -- (0,1.2) node[above] {$y$};
        
        % 绘制一个圆
        \draw[zillTeal, thick] (0,0) circle (1);
        
        % 绘制一条射线
        \draw[zillOrange, thick] (0,0) -- (45:1) node[above right] {$P$};
        
        % 标注角度:注意 \theta 必须在 $ $ 内
        \draw (0.3,0) arc (0:45:0.3);
        \node at (0.5,0.2) {$\theta$}; 
    \end{tikzpicture}
}{单位圆}

对于单位圆上的点 $P(x,y)$,其横坐标即为 $\cos\theta$,纵坐标即为 $\sin\theta$。由此可见,当点 $P$ 在不同象限运动时,三角函数值的正负号会随之改变。

\begin{properties}
    \begin{enumerate}
        \item $\sin^2\theta+\cos^2\theta=1$
        \item \textbf{诱导公式}:奇变偶不变,符号看象限
    \begin{itemize}
    \item $\displaystyle \begin{aligned}[t] % [t] 表示顶部对齐
        & \sin(\theta + 2k\pi) = \sin\theta, && \cos(\theta + 2k\pi) = \cos\theta && \text{(周期性)} \\
        & \sin(-\theta) = -\sin\theta,       && \cos(-\theta) = \cos\theta       && \text{(奇偶性)}
    \end{aligned}$
    \item $\displaystyle \begin{aligned}[t]
        & \sin(\theta + \pi)=-\sin\theta,&&\cos(\theta+\pi)=-\cos\theta\\
        & \sin(\pi-\theta)=\sin\theta,&&\cos(\pi-\theta)=-\cos\theta
    \end{aligned}$
    \item $\displaystyle \begin{aligned}[t]
        & \sin(\frac{\pi}{2}-\theta)=\cos{\theta},&&\cos(\frac{\pi}{2}-\theta)=\sin\theta\\
        & \sin(\frac{\pi}{2}+\theta)=\cos{\theta},&&\cos(\frac{\pi}{2}+\theta)=-\sin\theta
    \end{aligned}$
\end{itemize}
    \item \textbf{和差化角}:
    $\displaystyle
    \begin{aligned}[t]
        \sin(a \pm b) &= \sin a \cos b \pm \cos a \sin b \\
        \cos(a \pm b) &= \cos a \cos b \mp \sin a \sin b
    \end{aligned}
    $
    \end{enumerate}

\end{properties}
\leftnote[-250pt]{
    在这些基本恒等式的基础上,我们也可以得到别的恒等式,例如二倍角公式和$\displaystyle \tan$有关的恒等式,这些恒等式我们会在例题中讨论
}
\par
    \begin{example}["拍"现象]
        生活中,两个频率相近的波发生干涉后会出现\kwd{拍}现象。例如钢琴调音时,调音师同时敲击待调的弦和标准弦,在听觉上会感到音量有周期性的强弱。
        试利用三角函数及其恒等变换式,来解释这种现象的原理。
        
        \kwd{note}:振动的数学表达式是:$\displaystyle A=\cos(\omega t+\phi)$



    \end{example}
% --- 讲义正文:三角函数的图像 ---

\subsubsection{三角函数的图像}

三角函数的图像通常被称为“正弦波形”(Sinusoidal waves)。在物理学中,它们是描述简谐运动、波动以及交流电最基本的数学模型。

\lefttikz[0pt]{
    \begin{tikzpicture}[>=stealth, scale=0.75]
        % 坐标轴
        \draw[->] (-0.5,0) -- (7,0) node[right] {\tiny $x$};
        \draw[->] (0,-1.5) -- (0,1.5) node[above] {\tiny $y$};
        % 绘制正弦曲线 (实线)
        \draw[zillTeal, ultra thick, domain=0:6.28, samples=100] plot (\x, {sin(\x r)});
        % 绘制余弦曲线 (虚线)
        \draw[zillOrange, thick, dashed, domain=0:6.28, samples=100] plot (\x, {cos(\x r)});
        % 标注关键点
        \node[below left] at (0,0) {\tiny $0$};
        \draw (1.57,0.1) -- (1.57,-0.1) node[below] {\tiny $\frac{\pi}{2}$};
        \draw (3.14,0.1) -- (3.14,-0.1) node[below] {\tiny $\pi$};
        \draw (6.28,0.1) -- (6.28,-0.1) node[below] {\tiny $2\pi$};
        \node[left] at (0,1) {\tiny $1$};
        \node[left] at (0,-1) {\tiny $-1$};
        % 图例
        \node[zillTeal, right] at (1,1.3) {\tiny $\sin x$};
        \node[zillOrange, right] at (4,1.3) {\tiny $\cos x$};
    \end{tikzpicture}
}{正弦与余弦图像}

\begin{properties}
    \textbf{正弦函数 $y = \sin x$ 的图像性质}
    \begin{itemize}
        \item \textbf{定义域与值域}:定义域为 $\mathbb{R}$,值域为 $[-1, 1]$。
        \item \textbf{周期性}:具有周期性,最小正周期 $T = 2\pi$。
        \item \textbf{对称性}:关于原点对称,是奇函数。
        \item \textbf{相位关系}:$\cos x = \sin(x + \frac{\pi}{2})$,即余弦图像可看作正弦图像向左平移 $\pi/2$。
    \end{itemize}
\end{properties}

\leftnote[150pt]{物理直觉:在物理中,我们更习惯研究一般形式:\\ $y = A \sin(\omega x + \phi)$ \\ 其中每一个参数都对应一个明确的物理量。}

\subsubsection{图像的变换与物理意义}

在物理竞赛中,理解函数图像的平移与伸缩如何对应物理状态的变化至关重要。

\begin{properties}
    \textbf{参数对图像的影响}
    \begin{itemize}
        \item \textbf{振幅 $A$}:决定图像在垂直方向的拉伸或压缩。在振动学中,$A$ 代表最大位移。
        \item \textbf{频率参数 $\omega$}:决定图像在水平方向的“胖瘦”。$\omega$ 越大,周期 $T = \frac{2\pi}{\omega}$ 越小,图像越拥挤。
        \item \textbf{初相位 $\phi$}:决定图像在水平方向的左右平移。
    \end{itemize}
\end{properties}
\leftnote[50pt]{
    对于 $y = \sin(\omega x + \phi)$,图像的平移量$\frac{\phi}{\omega}$不是$\phi$。
    必须先提取系数 $\omega$:
    \[ y = \sin\left[ \omega \left( x + \frac{\phi}{\omega} \right) \right] \]
    这说明图像是向左平移了 $\frac{\phi}{\omega}$ 个单位。在物理中,这意味着“初相位”与“时间偏移”是不同的概念。
}
\begin{example}[简谐振动方程解析]
    已知一个质点的振动方程为 $x(t) = 10 \sin(2\pi t + \frac{\pi}{3})$(单位:cm, s)。
    请回答:
    (a) 该振动的振幅 $A$ 和周期 $T$ 分别是多少?
    (b) 该图像是由标准正弦波 $y = 10 \sin(2\pi t)$ 向哪个方向平移了多少距离得到的?

    \begin{solution}
        (a) 由方程可知,振幅 $\displaystyle A = \qty{10}{cm}$。角频率 $\displaystyle \omega = 2\pi$,则周期 $\displaystyle T = \frac{2\pi}{\omega} = \frac{2\pi}{2\pi} = \qty{1}{s}$。
        
        (b) 提取系数 $\omega$:$\displaystyle 2\pi t + \frac{\pi}{3} = 2\pi (t + \frac{1}{6})$。
        根据“左加右减”原则,该图像是由 $y = 10 \sin(2\pi t)$ 向左平移了 $1/6$ 个单位距离得到的。
    \end{solution}
\end{example}

\subsubsection{反三角函数及其图像}

当我们需要从三角函数的值反推对应的角度时,就需要用到反三角函数。由于三角函数在整个实数域上不是单调的,为了定义其反函数,我们必须\kwd{限制其定义域},使其在特定区间内是“一一对应”的。

\leftnote[-50pt]{\textbf{物理意义}:\\
反三角函数的结果是一个\textbf{角度}。在物理竞赛中,它最常用于通过力或速度的分量来确定矢量的方向(即求偏角)。}

\begin{definition}{反正弦与反正切函数}
    \begin{itemize}
        \item \kwd{反正弦函数}: $y = \arcsin x$:是 $x = \sin y$ 在 $y \in [-\frac{\pi}{2}, \frac{\pi}{2}]$ 上的反函数。
        \item \kwd{反正切函数}: $y = \arctan x$:是 $x = \tan y$ 在 $y \in (-\frac{\pi}{2}, \frac{\pi}{2})$ 上的反函数。
    \end{itemize}
\end{definition}

\begin{properties}
    \textbf{常用反三角函数的定义域与值域}
    \par\medskip
    在计算时,务必注意结果(角度)的取值范围:
    \begin{itemize}
        \item $y = \arcsin x$:定义域 $[-1, 1]$,值域 $[-\frac{\pi}{2}, \frac{\pi}{2}]$。
        \item $y = \arccos x$:定义域 $[-1, 1]$,值域 $[0, \pi]$。
        \item $y = \arctan x$:定义域 $\mathbb{R}$,值域 $(-\frac{\pi}{2}, \frac{\pi}{2})$。
    \end{itemize}
\end{properties}
\lefttikz[-100pt]{
    \begin{tikzpicture}[>=stealth, scale=1.3]
        % 1. 坐标轴
        \draw[->] (-1.5,0) -- (1.5,0) node[right] {\tiny $x$};
        \draw[->] (0,-1.8) -- (0,1.8) node[above] {\tiny $y$};
        
        % 2. 绘制 y = sin x (背景虚线)
        \draw[gray, dashed, domain=-1.57:1.57, samples=50] plot (\x, {sin(\x r)});
        
        % 3. 绘制 y = x (对称轴)
        \draw[gray, ultra thin] (-1.5,-1.5) -- (1.5,1.5);
        
        % 4. 核心修正:绘制 y = arcsin x
        % 虽然数学上是 x=sin(y),但在 TikZ plot 里我们仍用 \x 作为遍历变量
        \draw[zillOrange, ultra thick, domain=-1.57:1.57, samples=50] plot ({sin(\x r)}, \x);
        
        % 5. 标注
        \node[below left] at (0,0) {\tiny $0$};
        \node[zillOrange, right] at (0.7, 1.3) {\tiny $y = \arcsin x$};
        \node[left] at (0, 1.57) {\tiny $\pi/2$};
        \node[left] at (0, -1.57) {\tiny $-\pi/2$};
        \draw (1,0.05) -- (1,-0.05) node[below] {\tiny $1$};
        \draw (-1,0.05) -- (-1,-0.05) node[below] {\tiny $-1$};
    \end{tikzpicture}
}{反正弦函数图像与其对称性}



\begin{caution}{符号的使用习惯}
    在许多物理教材和科学计算器中,反正弦函数常记作 $\sin^{-1} x$。
    请务必记住:
    \[ \sin^{-1} x \equiv \arcsin x \neq \frac{1}{\sin x} \]
    指数 $-1$ 在这里表示“反函数”,而非“倒数”。
\end{caution}

\begin{example}[利用分量确定矢量方向]
    在平面直角坐标系中,一个力矢量的水平分量为 $F_x = \qty{3.0}{N}$,竖直分量为 $F_y = \qty{4.0}{N}$。试求合力 $\bm{F}$ 与 $x$ 轴正方向的夹角 $\theta$。
    
    \begin{solution}
        根据矢量合成的几何关系,合力的方向角 $\theta$ 满足正切关系:
        \[ \tan \theta = \frac{F_y}{F_x} = \frac{4.0}{3.0} \approx 1.333 \]
        为了求出角度 $\theta$,我们对等式两边取反正切函数:
        \[ \theta = \arctan\left( \frac{4.0}{3.0} \right) \]
        利用计算器计算可得:
        \[ \theta \approx 0.927 \text{ rad} \quad (\text{约 } 53.1^\circ) \]
        这说明该力指向第一象限,与水平方向成 $53.1^\circ$ 角。
    \end{solution}
\end{example}

\leftnote[-130pt]{在物理竞赛的复杂计算中,我们往往不需要算出具体的角度数值,直接保留 $\theta = \arctan(2/3)$ 这种形式通常也是被接受的。}
\begin{exercises}{1.3.1 }

    \noindent\textit{在问题 1-5 中,探讨正切公式与二倍角公式的代数变形}
    \begin{exerciseList}
        \item \kwd{正切展开}:已知 $\tan \alpha = 1/2$,$\tan \beta = 1/3$,利用和角公式计算 $\tan(\alpha + \beta)$ 的值,并指出 $\alpha + \beta$ 在 $(0, \pi)$ 范围内的具体弧度。
        \item \kwd{斜面倾角}:在物理受力分析中,经常遇到 $\tan(\theta + 45^\circ)$。请将其展开为仅含 $\tan \theta$ 的表达式。
        \item 利用二倍角公式证明:$\frac{1 - \cos 2\theta}{\sin 2\theta} = \tan \theta$。这个公式常用于将复杂的振动方程简化。
        \item \kwd{降幂公式}:利用 $\cos 2\theta$ 的变形公式,将 $\sin^2 \theta$ 和 $\cos^2 \theta$ 分别表示为含 $\cos 2\theta$ 的一次表达式。这在计算交流电有效值时非常重要。
        \item 已知 $\tan \theta = t$,请利用倍角公式推导出 $\tan 2\theta$ 随 $t$ 变化的表达式,并讨论当 $\theta \to \pi/4$ 时,该式的变化趋势。
    \end{exerciseList}

    \noindent\textit{在问题 6-10 中,掌握万能公式 (Universal Substitution) 的应用。}
    \begin{exerciseList}
        \item \kwd{万能公式推导}:设 $t = \tan(\theta/2)$,利用倍角公式和 $\sin^2 + \cos^2 = 1$ 证明:
        \[ \sin \theta = \frac{2t}{1+t^2}, \quad \cos \theta = \frac{1-t^2}{1+t^2} \]
        \item \kwd{参数表达}:若一个物体的速度分量为 $v_x = v_0 \cos \theta, v_y = v_0 \sin \theta$,请利用 $t = \tan(\theta/2)$ 将其坐标分量转化为关于 $t$ 的分式方程。
        \item 利用万能公式求解方程:$\sin \theta + \cos \theta = 1$。
        \item 证明:$\frac{\sin \theta}{1 + \cos \theta} = \tan\frac{\theta}{2}$。该结论在光学中分析半偏角时经常使用。
        \item 思考:万能公式为什么被称为“万能”?在处理含有 $\sin$ 和 $\cos$ 的复杂分式方程时,这种代换有什么优势?
    \end{exerciseList}

    \noindent\textit{在问题 11-15 中,结合反三角函数进行深度运算。}
    \begin{exerciseList}
        \item 计算 $\tan(\arcsin \frac{3}{5})$ 的值(提示:先设 $\theta = \arcsin \frac{3}{5}$,画出直角三角形)。
        \item 证明反正切加法公式:$\arctan x + \arctan y = \arctan \frac{x+y}{1-xy}$(假设 $xy < 1$)。
        \item \k w d{偏角计算}:已知 $\bm{v} = (1, t)$,利用 $\theta = \arctan t$ 表达速度方向,并推导当 $t$ 翻倍时,$\tan \theta$ 如何变化。
        \item 计算 $\sin(2 \arctan \frac{1}{3})$。请直接利用万能公式的结果。
        \item 求解 $x$:$\arctan(x) + \arctan(2x) = \frac{\pi}{4}$。
    \end{exerciseList}

    \noindent\textit{在问题 16-20 中,解决具有竞赛背景的物理综合问题。}
    \begin{exerciseList}
        \item \kwd{摩擦角模型}:一物体放在粗糙斜面上,恰好要下滑时,重力的下滑分量等于最大静摩擦力,即 $mg\sin \theta = \mu mg\cos \theta$。
        \begin{enumerate}
            \item 证明此时斜面倾角 $\theta = \arctan \mu$($\mu$ 为摩擦系数)。
            \item 若斜面倾角变为 $\theta + \phi$,请用 $\tan \theta$ 和 $\tan \phi$ 表示此时合力倾向的切向比例。
        \end{enumerate}
        \item \kwd{斜抛射程优化}:已知射程 $R = \frac{v_0^2}{g} \sin 2\theta$。若要在水平距离 $R$ 处击中高为 $H$ 的目标,需满足 $H = R \tan \theta - \frac{gR^2}{2v_0^2 \cos^2 \theta}$。利用 $1/\cos^2 \theta = 1 + \tan^2 \theta$,将此方程转化为关于 $\tan \theta$ 的一元二次方程。
        \item \kwd{光的折射近似}:当光线从空气射入水中($n=1.33$),在入射角 $\theta$ 很小时,利用 $\tan \theta \approx \sin \theta \approx \theta$,证明实际深度 $h$ 与视深 $h'$ 的关系为 $h/h' \approx n$。
        \item \kwd{振动合成}:两个同方向振动 $x_1 = A\cos(\omega t)$ 和 $x_2 = A\cos(\omega t + \phi)$ 叠加。利用和差化积(或和角公式展开)证明合振幅 $A_{total} = 2A \cos(\phi/2)$。
        \item \kwd{视角问题}:一个高为 $L$ 的物体竖直立在地面。观察者眼睛离地高度为 $h$,水平距离为 $x$。请用 $\arctan$ 写出观察者观察该物体时,眼睛张开的视角 $\gamma$ 的解析式。
    \end{exerciseList}
\end{exercises}

\subsection{指数函数}
\begin{example}[银行复利]
\end{example}

 
\end{document}